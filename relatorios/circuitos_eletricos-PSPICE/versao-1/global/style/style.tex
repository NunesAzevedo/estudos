\geometry{
  left   = 2.5 cm, % Margem esquerda
  right  = 2.5 cm, % Margem direita
  top    = 3   cm, % Margem superior ajustada para incluir o cabeçalho
  bottom = 4   cm  % Margem inferior
}

% ---------------------------------
%   Configurações do cabeçalho
% ---------------------------------

\pagestyle{fancy}
\fancyhf{}

% Logo da USP
\fancyhead[L]{\includegraphics[width = 3 cm]{./assets/images/logo/logo-usp.jpg}} 

\fancyhead[C] {\textbf{Universidade de São Paulo \\ Escola de Engenharia de São Carlos \\ Instituto De Ciências Matemáticas e de Computação} \\ Paulo Nunes de Azevedo} % Cambia la materia

% Logo da EESC
\fancyhead[R]{\includegraphics[width = 2 cm]{assets/images/logo/logo-eesc.png}  \\ \includegraphics[width = 2 cm]{assets/images/logo/logo-icmc.jpg}}
\fancyfoot[C]{\thepage}

\renewcommand{\headrulewidth}{1 pt} % Espessura da linha que quebra o cabeçalho
\renewcommand{\footrulewidth}{1 pt}

% Ajuste del espacio entre el encabezado y el texto
\setlength{\headsep}{1.9cm} % Espaço entre o cabeçalho e o início do texto

% Aplicar o cabeçalho na capa
% \fancypagestyle{plain}{
%   \fancyhf{}

%   % Logo da Universidade
%   \fancyhead[L]{\includegraphics[width=1.3cm]{assets/images/logo/logo.jpg}}
%   \fancyhead[C]{\textbf{Unidad Profesional Interdisciplinaria de \\Ingeniería Campus Tlaxcala} \\ Dr. Jesús García Ramírez \\ Materia} % Cambia este texto por el que desees

%   % Logo da outra instituição
%   \fancyhead[R]{\includegraphics[width=3cm]{assets/images/logo/UPIIT.jpg}}
%   \fancyfoot[C]{\thepage}
%   \renewcommand{\headrulewidth}{1pt}
%   \renewcommand{\footrulewidth}{1pt}
% }

\renewcommand{\refname}{Referencias}


\title{Nome da Disciplina \#1}
\author{Nombre del Estudiante(s)}
\date{\date{\today}}
